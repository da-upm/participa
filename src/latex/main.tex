%%%%%%%%%%%%%%%%%%%%%%%%%%%%%%%%%%%%%%%%%%%%%%%%%%%%%%%%%%%%%%%%%%%%%%%%%%%%%%
%
% Delegación de Alumnos de Telecomunicación
% PLANTILLA DE DOCUMENTOS EN LaTeX
%
% Esta plantilla consiste en la plantilla oficial de documentos de DAT,
% creada originalmente en LibreOffice y exportada a PDF, poniéndola como
% fondo en un documento de LaTeX estándar con algunas modificaciones extra
% (fuentes, márgenes, etc.).
%
%%%%%%%%%%%%%%%%%%%%%%%%%%%%%%%%%%%%%%%%%%%%%%%%%%%%%%%%%%%%%%%%%%%%%%%%%%%%%%

\documentclass[11pt]{article} % Documento de tipo "artículo", con fuente por defecto de 11pt.

\usepackage{amsmath}

%\usepackage{fontspec}
%\newfontfamily\comfortaa{Comfortaa}[
%    Extension=.ttf,%
%    Ligatures={TeX,Common},%
%    FontFace={l}{n}{*-Light},%
%    FontFace={l}{it}{Font=*-Light,FakeSlant=0.167},%
%    UprightFont={*-Regular},%
%    ItalicFont={*-Regular},%
%    ItalicFeatures={FakeSlant=0.167},%
%    FontFace={b}{n}{*-Bold},%
%    FontFace={b}{it}{Font=*-Bold,FakeSlant=0.167},%
%    BoldFont={*-Bold},%
%    BoldItalicFont={*-Bold},%
%    BoldItalicFeatures={FakeSlant=0.167}]
 

\usepackage[default]{comfortaa}
\usepackage[T1]{fontenc}      % Usa fuentes "ricas", que facilitan el copy-paste en la salida.
\usepackage[spanish]{babel}   % Traducción al español de meses, funciones matemáticas, etc.
\usepackage{enumitem}         % Ajuste de espacios entre elementos de listas.
\usepackage{graphicx}         % Permite la inserción de imágenes.
\usepackage{hyperref}         % Convierte las referencias en hiperenlaces y añade el comando \url.
\usepackage{wallpaper}        % Para la plantilla del fondo.
\usepackage{xcolor}           % Fuentes de colores.
\usepackage{fbb}              % Fuente para documentos.
\usepackage[                  % Márgenes personalizados.
  a4paper,
  lmargin=2cm,
  rmargin=2cm,
  tmargin=4.5cm,
  bmargin=3.5cm,
  headheight=3cm
]{geometry}
\usepackage{setspace}

\usepackage[]{hyphenat}   % Evita el corte de palabras cuando no caben en una unica linea.

% Fondo de la plantilla de DAT.
\ULCornerWallPaper{1}{background_DA_general_X.pdf}

% Elimina el sangrado de cada párrafo.
\setlength{\parindent}{0cm}
% Espacio de 1 línea entre párrafos.
\setlength{\parskip}{\baselineskip}

% Elimina cabeceras/pies de página por defecto.
\pagestyle{empty}

% Elimina la separación adicional entre ítems de listas.
\setlist[enumerate]{itemsep=0mm}

% Definición de colores de la plantilla.
\definecolor{DAblue}{HTML}{00509B}

% Para hacer pruebas
\usepackage{lipsum}

% Nuestras definiciones de tipos de párrafos
\newcommand{\titulo}{\centering\fontsize{20}{13.5}\textcolor{DAblue}\bfseries\comfortaa}

\newcommand{\subtitulo}{\raggedright\comfortaa\fontsize{16}{18.5}\bfseries}

\newcommand{\subtituloazul}{\raggedright\comfortaa\fontsize{16}{18.5}\bfseries\textcolor{DAblue}}

\newcommand{\subtitulocentro}{\centering\comfortaa\fontsize{16}{18.5}\bfseries}

\newcommand{\titulillo}{\raggedright\comfortaa\fontsize{13}{15}\bfseries}

\newcommand{\titulilloazul}{\raggedright\comfortaa\fontsize{13}{15}\bfseries\textcolor{DAblue}}

\newcommand{\cuerpo}{\raggedright\comfortaa\fontsize{11}{13.5}\fontseries{l}\selectfont}

\newcommand{\resaltado}{\raggedright\comfortaa\fontsize{11}{13.5}\fontseries{l}\selectfont\textcolor{DAblue}}

\begin{document}


\titulo
\textbf{Propuestas de la Comunidad Universitaria y la Delegación de Alumnos de la UPM para las candidaturas a rector/a 2024}

\textbf{\textcolor{DAblue}}

\vspace{2cm}

\section*{\subtituloazul{Introducción}}

\cuerpo{
    Desde la Delegación de Alumnos de la Universidad Politécnica de Madrid (DA-UPM) y con
    motivo de las elecciones a rector/a de la UPM de este 2024, se ha realizado una recogida de
    propuestas a través de nuestra plataforma web \href{https://participa.da.upm.es}{participa.da.upm.es},
    por parte de todos los miembros de la comunidad universitaria en pro de la mejora del funcionamiento y el futuro de
    la Universidad.
}

\cuerpo{
    Tras dicha recogida de propuestas, se sometieron a una votación abierta a todos los miembros de nuestra comunidad universitaria. Estas se presentarán a todos los 
    candidatos, trasladando de esta manera el sentir de nuestros compañeros y compañeras en una lista de objetivos. Tras un proceso de negociación, los candidatos se podrán 
    comprometer a cumplir las demandas que consideren a lo largo de los próximos 6 años de legislatura en caso de salir elegidos.
}

\cuerpo{
    Posteriormente, se procederá a la publicación de este documento y la respuesta recibida en las redes sociales, página web y en la plataforma destinada a las elecciones 
    a rector de la UPM del 2024, para uso y conocimiento de todos.
}

\cuerpo{
    A continuación, se presentan las propuestas apoyadas y sus matices o compromisos adquiridos por el candidato en caso de ser elegido/a rector/a de la UPM.
}

\newpage

\section*{\subtituloazul{Propuestas y compromisos}}

%CONTENT%

\newpage

\section*{\subtituloazul{Conclusiones}}

\cuerpo{
    La participación activa de todos los colectivos que integran la Universidad Politécnica de Madrid en los procesos de toma de decisiones es fundamental de cara a la 
    mejora, desarrollo y evolución de la propia universidad. Para que esto sea posible, es imprescindible que se realice un proceso de diálogo entre todas las partes 
    implicadas, y que todas las demandas sean escuchadas y consensuadas
}

\cuerpo{
    Si se cumplen estas premisas, tarde o temprano, todos los participantes del proceso acaban sintiéndose parte del mismo, convirtiéndose de esta manera en un eslabón en 
    la cadena de valor de la Universidad, y fomentando su mejora, desarrollo e innovación. Es por ello que nosotros, como representantes de nuestros compañeros, 
    consideramos que este documento es nuestro aporte para alcanzar el objetivo de una Universidad Pública accesible y de calidad, siendo las siguientes medidas las que 
    consideramos fundamentales para poder alcanzar este objetivo.
}

\section*{\subtituloazul{Compromiso}}

\cuerpo{
    Mediante este documento, los candidatos a rector/a de la UPM adquieren un compromiso con la comunidad universitaria de la UPM, comprometiéndose a cumplir con las propuestas seleccionadas en los términos indicados por ellos mismos en caso de resultar elegidos. 
}

\cuerpo{
    Una vez los candidatos hayan cumplimentado y firmado electrónicamente este documento, la delegación se compromete a su publicación a través de su página web y redes sociales, indicando cuáles de los candidatos que se presentan a rector/a de esta Universidad han decidido hacer suyas las propuestas recogidas en este documento.
}

\begin{center}
    \begin{figure}[h]
        \centering
        \includegraphics[width=0.5\textwidth]{logoRector.png}
    \end{figure}
    Firmado digitalmente por el candidato \\
    \textcolor{DAblue}{
      %CANDIDATE% \\
    }
\end{center}

\end{document}